\epigraph{Cita}{Autor}

\section{Sección uno}\label{seccion:SeccionUnoCapituloUno}

Las sociedades~\cite{citaEjemplo23} \dots
	
\section{Sección dos}\label{seccion:SeccionDosCapituloUno}

Una de \dots

\begin{itemize}
	\item \textbf{Uno}: Los \dots
	\item \textbf{Dos}: Los \dots
	\item \textbf{Tres}: Los \dots
	\item \textbf{Cuatro}: Los \dots
	\item \textbf{Cinco}: Los \dots
\end{itemize}

\section{Sección tres}\label{seccion:SeccionTresCapituloUno}

El acrónimo uno (\acsp{AO}, por sus siglas en inglés) \dots

\subsection{Subsección uno}\label{subsection:SubseccionUnoSeccionTresCapituloUno} 

\subsection{Subsección dos}\label{subseccion:SubseccionDosSeccionTresCapituloUno}

Las \dots

La Figura~\ref{figura:EjemploUno}
\begin{figure}[!ht]
	\centering
	\includesvg[width=7cm]{imagenes/logoUGR.svg}
	\caption{Figura de ejemplo}
	\label{figura:EjemploUno}
\end{figure}

\iffalse % Para guardar texto redactado como comentario para su posterior reutilización
\begin{itemize}
	\item \textbf{Ejemplo uno}:  
	\item \textbf{Ejemplo dos}: 
	\item \textbf{Ejemplo tres}: 
\end{itemize}
\fi

\subsubsection{Subsubsección uno}\label{subsubseccion:SubsubseccionUnoSubseccionUnoSeccionTresCapituloUno}

\subsubsection{Subsubsección dos}\label{subsubseccion:SubsubseccionDosSubseccionUnoSeccionTresCapituloUno}

\subsubsection{Subsubsección tres}\label{subsubseccion:SubsubseccionTresSubseccionUnoSeccionTresCapituloUno}

\subsection{Subsección tres}\label{subseccion:SubseccionTresSeccionTresCapituloUno} 

\subsubsection{Subsubsección uno}\label{subsubseccion:SubsubseccionUnoSubseccionTresSeccionTresCapituloUno}